\section{Lýsing}

\begin{center}
Þessi róboti mun nota two sonic sensora til að aka áfram þar til að hann kemur að hlut, þá mun hann mæli hvað hluturinn er stór og eftir stærð hans mun róbotinn ákveða hvort hann tekur hlutinn upp, ýtir honum, eða beygir og heldur áfram að aka.
Að lokum mun hann koma aftur þar sem hann byrjaði.

Verður allt þetta forritað í Vex, þar sem það var okkar skoðun að Vex er léttara að nota en Arduino.


Í staðinn fyrir fjögur hjól eru þrjú, tvö á hliðunum og eitt aftan á sem gerir honum auðveldara fyrir að gera krappar beygjur og einnig fyrir betri fínhreyfingar, þeim er stjórnað með þremur Vex móturum, einnig í stað venjulegra hjóla notum við bumper hjól sem eru meira stöðug þegar þau keyra yfir ójafnt yfirborð sem henta áætlunum okkar mun meira en hin venjulegu.
Krananum er stjórnað með tveimur móturum, einum sem stjórnar arminum sem fer upp og niður og öðrum sem stjórnar klónni.
Það verður líka hægt að stjórna honum bara með fjarstýringu þar maður getur bara gert það sem maður will.
Sonic Sensorarnir tveir eru notaðir til að kanna ummálið á hlutum fyrir framan þig, sem ákveður hvort að eigi að taka hann up eða ekki


Ef hann ákveður að hluturinn sem getur verið hvað sem er til dæmist steinn, leikfang eins of action fígúra eða bara pilluglas þá lækkar hann arminn, keyrir áfram þar til að klóin umlýkur sagðan hlut og tekur hann upp .
Að því loknu snýr róbótinn til baka þar sem hann byrjaði, lækkar arminn og sleppir hlutnum.
Hinsvegar ef hluturinn er of stór þá heldur róbótinn áfram meðfram aðskotahlutnum að vegginum þar til að hann finnur hlut sem er passlega stór.



Róbotar eins og þessir geta verið notaðir í ýmislegt eins og ef maður hefur tapað eitthverju á þröngum stað eða ofan á þunnum ís sem maður mundir ekki vilja að stíga á.
Þar getur maður sent róbótann þar sem hann er minni, léttari og kemst auðveldlega á staði sem flest fólk ætti í stökustu erfiðleikum að komast á.
Til dæmis gæti maður sent hann í helli eða hrunda byggingu, einfaldlega of dimmar aðstæður. eða skortir súrefni eða eiturefni hafa byrjað í andrúmsloftið sem maður gæti ekki andað að sér og þarf að komast þar inn till að kanna aðstæður.
Einnig getur maður notað hann til að mæla herbergi áður enn maður byrjar framkvæmdir, eins og þegar maður ætlar að leggja parket eða teppi.
Það væri líka einfaldlega hægt að nota róbótan sem leikfang, nota kranan til að taka up kubba eða aðra hluti og stakka þeim, búa til að völundarhús til að sjá hversu góður maður er að stýra honum, gá hvort hann komist yfir ýmsar torfærur og svo framvegis.

\end{center}